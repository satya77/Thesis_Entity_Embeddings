\begin{center}
  \textsc{Abstract}
\end{center}
%
\noindent
%
Because of the technological advances of recent years and vast amount of textual content being released every minute, natural language processing has grown into an exciting area of scientific research. With the goal of learning and understanding human language content has become more significant.
% BR: Split up and make everything from 'With the goal' into a second
% sentence.
Since a \emph{``word''} is single distinct meaningful element of speech or sentence, capturing its semantics plays an important rule for understanding human language.
Word embeddings are a vector representation of words, which map the words of a high-dimensional vocabulary to vectors of real numbers in a low-dimensional space, such that words with similar meaning are mapped to nearby points, measured in terms of the Euclidean distance, for example.
% example
% BR: add '..., measured in terms of the Euclidean distance, for
% example'.
Word embeddings are used as features in many information retrieval and natural language processing tasks. However, while many such tasks involve or even rely on named entities, popular word embedding models so far fail to recognize them as a distinct concept of their own and rather treat all words equally .
% BR: What is a term in this context? Do you want to add an explanation
% in the form of 'as terms, i.e. <EXPLANATION>'?
Although it seems intuitive that applying word embedding techniques to a corpus annotated with named entities should result in more intelligent word features, this naive approach results in a degraded performance in comparison to embeddings trained on raw, unannotated text.
% BR: re-arrange: 'this naive approach results in a degraded performance
% in comparison to...'
Moreover, annotating the corpus increases the complexity of the relationship between words. Different named entity types have a different impact on the word semantics, which normal word embeddings fail to capture.
% BR: re-arrange: 'increases the complexity of the relationship between words'
% BR: add 'a': 'have _a_ different impact'
\\
In this thesis, we propose novel approaches to jointly train word and entity embeddings on a large corpus with automatically annotated and linked entities.
Furthermore, we extend our approaches to capture the relation of a word to a specific types of entities in separate components, which creates a more interpretable representation and shines some light on the significance of entity types for a word semantics.
% BR: 'shine some light _on_ the significance'
% BR: I would split up this sentence, maybe after 'components': This
% creates a more interpretable representation...'
We discuss two approaches for training entity embeddings, namely training of state-of-the-art word embeddings techniques on annotated text, as well as embedding the nodes of a co-occurrence graph representation of an annotated corpus. We take advantage of this graph structure to introduce the embeddings with separable components, where each component captures the relation of the word to named entities of a specific type. To obtain these separable components, we modify well-established embeddings for graphs and words to divide the embedding space into types of the named entities present in the text. \\
% BR: 'To achieve these' --> 'To obtain these'
% BR: I would also rather rephrase the embeddings: '...we modify
% well-established embeddings for graphs and words...'
Finally, we compare the performance of our approaches against classical word embeddings trained on the raw text on a variety of word similarity, analogy, and clustering evaluation tasks. We furthermore explore the possible benefits and use-cases of such  models with an entity-specific experimental analysis. 
% BR: Add 'Finally, we compare...'
% BR: Write 'classical word embeddings' rather than 'the classical word
% embeddings'
% BR: Use 'furthermore' rather than 'further' :-)
% BR: Write 'models with _an_ entity-specific' 

%annotating the corpus adds more complexity to data, since the word is no longer just terms but have a specific type (e.g., location or actor). 
%Named entities are real-world objects, such as actors and locations, that can be denoted with a proper name and fall into pre-defined categories.
% in an annotated corpus, where the type (e.g., location or actor) of each word is known, studying the different named entity types separately and investigating their impact on word semantics is also a possibility. 

%We find that simply training popular word embedding models on an annotated corpus is not enough to achieve an acceptable performance and discuss how and when node embeddings of the co-occurrence graph can restore the performance. We also conclude that dividing the embedding space into separate components, requires a more complex definition of a word's context and simple approaches are not sufficient to compete with normal word embedding on common evaluation tasks. The study of the separate component suggests that the terms in a text are the most important learning input and removing them or restricting the algorithms to other types degrades the performance on common tasks.
