%%%%%%%%%%%%%%%%%%%%%%%%%%%%%%%%%%%%%%%%%%%%%%%%%%%%%%%%%%%%
\chapter{Conclusions}\label{chap:concl}
In this thesis, we investigated different techniques to jointly train embeddings for terms and entities on an annotated corpus with named entities. We considered the naive application of popular models, namely, word2vec and GloVe, to annotated texts, as well as embedding nodes of a co-occurrence graphs extracted from the annotated text. We used popular node embedding methods DeepWalk and VERSE and modified them to obtain entity embeddings. Furthermore, to analyse the effect of each entity type on the embedding space, we explored variations of GloVe, word2vec, and DeepWalk that separate the embedding space into different embedding types. These faceted models consist of separable components, where each component indicates the relation of a word to entities of a specific type and can be used for type-specific analysis. Moreover, we compared all proposed models to traditional word embeddings on a comprehensive set of term-focused evaluation tasks and performed entity-centric exploration to identify strengths and weaknesses of each approach. 
\section{Discussion}
We found that even though training state-of-the-art word embeddings directly on annotated text is possible, their performance degrades in term-centric tasks, particularly in tasks that depend on relatedness. By contrast, graph-based embeddings, while bad at capturing similarity, perform better on the relatedness tasks. Moreover, they provide more insights into entity-entity relations and thus, work better for entity-centric problems. Graph-based embeddings, in particular, show great promise for entity-centric analogy tasks, but fail to form meaningful clusters of words, whereas normal word embeddings are able to cluster similar words and even multi-word entities better and perform well on purely term-based analogy tasks. Exploration of entity neighbourhoods for each model shows that proximity of words with similar meanings is better for regular word embeddings, as the neighbourhood shows only synonymous or descriptive words.
% BR: The preceding sentence has an issue with grammar. Do you want to
% say that the proximity of words with similar meanings is better for
% regular word embeddings?
On the other hand, models on annotated text and graph-based embeddings, in particular, tend to map related and associated entities closer in embedding space and can be used for searching for related entities. In general, the usefulness of entity embeddings is very dependent on the problem at hand, and should not be used blindly in place of word embeddings.
\\
Since faceted models use annotated text as input, they share many characteristics with the entity embeddings. Faceted models tend to favour relatedness over similarity and show promising results in entity-centric analogy tasks. However, dividing the embedding space by removing certain entity types during training has its drawbacks. The performance significantly degrades in comparison to normal word embeddings on raw text, which we consider due to a lack of information to learn from. To train each component, the vocabulary is limited to a certain entity type, which limits the context of the word and makes it more difficult for the model to learn a useful representation. Approaches based on neural networks, however, require larger amounts of data and their performance is strongly coupled to data availability. Hence, limiting the context of a word to a certain type of entity reduces the training data set, resulting in a lower performance.
% BR: Rephrase: 'Approaches based on neural networks, however, require
% larger amounts of data and their performance is strongly coupled to
% data availability. Hence, limiting the context of a word to a certain
% type of entity reduces the training data set, resulting in a lower
% performance.'
We hypothesize that with a more involved definition of context for these models, where the limitation of context does not limit the data for the model to learn from, we might observe a boost in performance.
% BR: Rephrase: 'We hypothesize that with a more involved definition of
% context, ... we might observe observe'.
Despite the decrease in performance, faceted models show promise in entity-centric search, where each individual components can be used to limit the search space to a specific type. Moreover, similar to entity embeddings, they tend to map related entities closer together rather than synonymous words. Furthermore, by independent study of each component of faceted models, we found that surrounding terms are most significant when it comes to term-based relatedness and similarity tasks, while for clustering similar entities the component corresponding to the entity type tends to produce better results.

\section{Future Work}
Although the comparison of entity-based models with word embeddings on term-based datasets sheds some light onto their differences and potential use-cases, true performance of these models can only be evaluated if entity-centric evaluation datasets are created.
% BR: Phrasing: 'sheds some light...'
As future work, creating a dataset that contains named entities for common intrinsic tasks for word embedding and designing specific intrinsic tasks for entity-centric evaluation will support the evaluation of future entity-based models to efficiently evaluate their methods.\\
% BR: replace 'can help' with 'will support the evaluation...'
Because of the ability of entity-based models to capture relatedness, a temporal analysis of the evolution of entity-entity relations in a dataset with a temporal aspect, such as news streams, is possible. Similar approaches have been used to study the transition in meaning for a word over time, which can be utilized to investigate the evolution of entity-entity relations. Moreover, separate components of faceted embeddings allow for a more detailed study, where the relation of an entity to associated locations or actors are analysed separately. \\
Overall, considering all weaknesses and strength of faceted and entity-based models, we see potential applications for entity embeddings and faceted models in entity-centric tasks that benefit from relatedness relations, such as improved query expansion based on related entities. Document ranking can also benefit from a vector representation of entity and words, especially when the document contains related entities to query words. Faceted models can be used to create more flexible search criteria based on entity types, which is beneficial for entity ranking and search engines. Tasks that use named entities as part of their pipeline can also benefit from entity embeddings, such as named entity recognition and disambiguation, which we consider to be the most promising future research directions and downstream tasks.\\
